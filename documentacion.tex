\documentclass[10pt]{article}
\usepackage[utf8]{inputenc}
\usepackage[spanish]{babel}
\usepackage{listings}
\usepackage{tcolorbox}
\usepackage{fancyvrb}
\usepackage{float}
\usepackage{amsmath}
\usepackage{amsfonts}
\usepackage{multirow}
\usepackage{multicol}
\usepackage{rotating}
\usepackage[hidelinks]{hyperref}
\setlength{\parskip}{2mm}
\usepackage{geometry}
\usepackage{siunitx}
\usepackage{tikz}
\usepackage{pgfplots}
\usepackage{graphicx}
\pgfplotsset{compat=newest}
\usepgfplotslibrary{units}
\let\olditemize\itemize
\def\itemize{\olditemize\itemsep=-1pt }

\geometry{legalpaper, margin=1.2in}
\linespread{1.4}
\setlength{\marginparwidth}{3.5cm}
\setlength{\parindent}{0cm}

\begin{document}

\begin{titlepage}

\newcommand{\HRule}{\rule{\linewidth}{0.5mm}}
\center
\textsc{ }\\[9cm]

\HRule \\[0.8cm]
{\LARGE PRÁCTICA DE BÚSQUEDA LOCAL}\\[1cm]
\textsc{INTEL·LIGÈNCIA ARTIFICIAL}\\
\textsc{Departament de Ciències de la Computació}\\[0.5cm]
\HRule \\[1cm]

\textsc{GRAU EN ENGINYERIA INFORMÀTICA}\\
\textsc{UNIVERSITAT POLITÈCNICA DE CATALUNYA}\\[9cm]

\vfill
\textsc{Mario Fernández - DNI} \\ [0.3cm]
\textsc{Héctor Fortuño - DNI}\\[0.3cm]
\textsc{Aina Luis - 26594595V}\\[0.3cm]
{\large Abril 2023}\\

\end{titlepage}

\pagestyle{empty}


\newpage
\tableofcontents
\newpage
\setcounter{page}{1}
\pagestyle{plain}
\section{Introducción}
El cambio climático nos esta obligando a reducir las emisiones de CO$_{2}$. Un primer paso para conseguirlo es reducir el uso del transporte privado, fomentando iniciativas de compartición de coches entre personas. Actualmente ya existen diversas plataformas que lo facilitan y ayudan a reducir el tráfico, el consumo de combustibles y la emisión de contaminantes. Así pues, el objetivo principal de esta práctica es optimizar un sistema de compartición de coches entre personas, haciendo uso de los algoritmos de Búsqueda Local de Hill Climbing y Simulated Annealing y las clases de la librería AIMA de Java.

En este informe se incluyen todas las decisiones que se han tomado para el buen desarrollo de la práctica. En primer lugar, se ha definido cuál es el problema que se quiere solucionar. Seguidamente, se han incluido tanto la representación como detalles de la implementación de los elementos del problema. Finalmente, se ha hecho una explicación de la experimentación que se ha realizado, así como la exposición de las conclusiones a las que se han llegado.

\newpage
\section{Definición del problema}

\newpage
\section{Representación y elementos del problema}
\subsection{Representación del estado}
\subsection{Operadores}
\subsection{Función heurística}
\subsection{Función generadora del estado inicial}

\newpage
\section{Implementación}

\newpage
\section{Experimentación}

\newpage
\section{Conclusiones}

\end{document}
